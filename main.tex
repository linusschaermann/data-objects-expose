\documentclass[12pt,oneside]{article}

%%%%%%%%%%%%%%%%%%%%%%%%%%%%
%%   Zusaetzliche Pakete  %%
%%%%%%%%%%%%%%%%%%%%%%%%%%%%
% use enumitem (do NOT load the older 'enumerate' package together with enumitem)
\usepackage{fancyhdr}
\usepackage{a4wide}
\usepackage{graphicx}
\usepackage{palatino}
\usepackage{multirow}
\usepackage{booktabs}
\usepackage{titlesec}
\usepackage{acronym}% http://ctan.org/pkg/acronym
\usepackage{enumitem}% http://ctan.org/pkg/enumitem

%folgende Zeile auskommentieren für englische Arbeiten
%\usepackage[ngerman]{babel}
%folgende Zeile auskommentieren für deutsche Arbeiten
\usepackage[ngerman, english]{babel}

\usepackage[T1]{fontenc}
\usepackage[utf8]{inputenc}
\usepackage[bookmarks]{hyperref}
\usepackage[justification=centering]{caption}
\usepackage[style=apa,natbib=true,backend=biber,maxbibnames=20]{biblatex}
\usepackage{csquotes}
\bibliography{literature}

\setlength{\parindent}{0em} 
\setlist[itemize]{noitemsep, topsep=0pt}
\setlist[enumerate]{noitemsep, topsep=0pt}
% Use numeric labels for enumerate levels:
\setlist[enumerate,1]{label=\arabic*.}
\setlist[enumerate,2]{label=\arabic{enumi}.\arabic{enumii}.}
\setlist[enumerate,3]{label=\arabic{enumi}.\arabic{enumii}.\arabic{enumiii}.}


%%%%%%%%%%%%%%%%%%%%%%%%%%%%%%
%% Definition der Kopfzeile %%
%%%%%%%%%%%%%%%%%%%%%%%%%%%%%%

\pagestyle{fancy}
\fancyhf{}
\cfoot{\thepage}
\setlength{\headheight}{16pt}

%%%%%%%%%%%%%%%%%%%%%%%%%%%%%%%%%%%%%%%%%%%%%%%%%%%%%
%%  Definition des Deckblattes und der Titelseite  %%
%%%%%%%%%%%%%%%%%%%%%%%%%%%%%%%%%%%%%%%%%%%%%%%%%%%%%

\newcommand{\JMUTitle}[9]{

  \thispagestyle{empty}
  \vspace*{\stretch{1}}
  {\parindent0cm
  \rule{\linewidth}{.7ex}}
  \begin{flushright}
    \vspace*{\stretch{1}}
    \sffamily\bfseries\Huge
    #1\\
    \vspace*{\stretch{1}}
    \sffamily\bfseries\large
    #2\\
    \vspace*{\stretch{1}}
    \sffamily\bfseries\small
    #3
  \end{flushright}
  \rule{\linewidth}{.7ex}

  \vspace*{\stretch{1}}
  \begin{center}
    \includegraphics[width=2in]{siegel} \\
    \vspace*{\stretch{1}}
    \Large Exposé  \\

    \vspace*{\stretch{2}}
   \large Lehrstuhl f\"{u}r Wirtschaftsinformatik\\
    \large und Systementwicklung\\
    \large Universität Würzburg\\
    \vspace*{\stretch{1}}
    \large Betreuer:  #8 \\[1mm]
    
    \vspace*{\stretch{1}}
    \large W\"urzburg, den #7 \\
        \vspace*{\stretch{0.25}}

    % Bearbeitungszeit: 14.03.2025 - 09.05.2025 % Die Bearbeitungszeit der Seminar-/ Abschlussarbeit ist hier einzutragen.

  \end{center}
}

\titlespacing*{\section}
{0pt}{3.5ex plus 1ex minus .2ex}{.2ex plus .2ex}
\titlespacing*{\subsection}
{0pt}{1.5ex plus 1ex minus .2ex}{.2ex plus .2ex}
\titlespacing*{\subsubsection}
{0pt}{1.5ex plus 1ex minus .2ex}{.2ex plus .2ex}




%%%%%%%%%%%%%%%%%%%%%%%%%%%%
%%  Beginn des Dokuments  %%
%%%%%%%%%%%%%%%%%%%%%%%%%%%%

\begin{document}

  \JMUTitle
      {Reconceptualizing Shadow IT in the Age of Generative AI: A Systematic Literature Review and Motivation Model of Shadow AI Usage}        % Titel der Arbeit
      {Linus Schärmann}                        % Vor- und Nachname des Autors
      {2910412}
      
      {Wirtschaftswissenschaftlichen Fakultät}  % Name der Fakultaet
      {W"urzburg 2025}                          % Ort und Jahr der Erstellung
      {15.02.2026}                              % Tag der Abgabe
      {Manuel Zall}               % Name des Erstgutachters
      {}                          % Name des Zweitgutachters

  \clearpage

\lhead{}
\pagenumbering{Roman} 
    \setcounter{page}{1}

\tableofcontents
\clearpage

%%%%%%%%%%%%%%%%%%%%%%%%%%%%
%%  Kurzzusammenfassung   %%
%%%%%%%%%%%%%%%%%%%%%%%%%%%%
% \newpage
% \lhead{Abstract}
% \section*{Abstract}
% \addcontentsline{toc}{section}{Abstract}

%\newpage
%\lhead{List of Figures} % Bei englischsprachiger Arbeit anzupassen auf: List of Figures
%\addcontentsline{toc}{section}{List of Figures} % Bei englischsprachiger Arbeit anzupassen auf: List of Figures
%\listoffigures

% \newpage
% \lhead{List of Tables} % Bei englischsprachiger Arbeit anzupassen auf: List of Tables
% \addcontentsline{toc}{section}{List of Tables} % Bei englischsprachiger Arbeit anzupassen auf: List of Tables
% \listoftables
% \newpage

\setlength{\parskip}{0.5em} 


%%%%%%%%%%%%%%%%%%%%%%%%%%%%%%%%%%
%%  Definition der Abkürzungen  %%
%%%%%%%%%%%%%%%%%%%%%%%%%%%%%%%%%%
\lhead{List of Abbreviations} % Bei englischsprachiger Arbeit anzupassen auf: List of Abbreviations
\section*{List of Abbreviations} % Bei englischsprachiger Arbeit anzupassen auf: List of Abbreviations
\addcontentsline{toc}{section}{List of Abbreviations} % Bei englischsprachiger Arbeit anzupassen auf: List of Abbreviations

\begin{acronym}
  \acro{IS}{Information Systems}
  \acro{SAI}{Shadow AI}
  \acro{SLR}{Systematic Literature Review}
\end{acronym}

%%%%%%%%%%%%%%%%%%%%%%%%%%%%
%%  Einstellungen  %%
%%%%%%%%%%%%%%%%%%%%%%%%%%%%
\clearpage
\pagenumbering{arabic}  
    \setcounter{page}{1}
\lhead{\nouppercase{\leftmark}}

%%%%%%%%%%%%%%%%%%%%%%%%%%%%
%%  Hauptteil  %%
%%%%%%%%%%%%%%%%%%%%%%%%%%%%

\section{Research Topic} \label{researchtopic}

Shadow IT refers to the unauthorized use of hardware, software or cloud services within an organization without approval from the IT department, potentially causing security risks, compliance issues or even data breaches. With the rise of generative AI technologies, such as large language models and AI-powered tools, a new category of shadow IT has emerged: Shadow AI. This term describes the use of generative AI applications by employees without formal approval or oversight from the organization's IT department. Shadow AI can lead to similar risks as Shadow IT, such as data privacy concerns, security vulnerabilities and compliance issues, but it also introduces unique challenges due to the nature of generative AI technologies.

\section{Motivation} \label{motivation}

The motivation for this thesis arises from the increasing prevalence of generative AI technologies in the workplace and the potential risks associated with their unauthorized use. While there is a growing body of research on Shadow IT, there is a lack of comprehensive understanding of Shadow AI and its implications for organizations. This thesis aims to fill this gap by conducting a systematic literature review to identify and analyze existing research on Shadow AI, and by developing a conceptual framework that extends the traditional understanding of Shadow IT to encompass the unique characteristics and implications of Shadow AI.

\section{Methodology} \label{methodology}

The research will follow a \ac{SLR} according to \citet{RN1}. This method is used to identify, evaluate and interpret the available research relevant to a particular research question, topic area or domain of interest and is designed to be used in \ac{IS} research \citep[884]{RN1}.

A \ac{SLR} is particularly suitable for this thesis as it allows to systematically gather and analyze existing knowledge on the topic of \acp{SAI}. With this approach, it is possible to identify research gaps and provide a solid foundation for future research directions. 

Even though the methodology is designed mostly for research teams, but with some minor adjustments, it can also be used for individual research. Overall it simplifies the process of conducting a literature review due to the lack of risks associated with incorrect team communication and cooperation.

\section{Preliminary Outline}

Lastly the following outline should give an overview of the planned sections of the thesis. Changes may appear during the research process but as an entry point it should help to structure the work.

\begin{enumerate}
  \item Introduction
  \item Theoretical Foundations
  \begin{enumerate}
    \item Shadow IT
    \item Motivational Theories
    \item From Shadow IT to Shadow AI
  \end{enumerate}
  \item Methodology
  \item Structured Literature Review Results
  \begin{enumerate}
    \item Overview of Identified Literature
    \item Shadow IT Baseline
    \item Shadow AI in Existing Literature
  \end{enumerate}
  \item Conceptual Development: Extending Shadow IT to Shadow AI
  \begin{enumerate}
    \item Transferable Motivations
    \item Novel Motivational Dimensions
    \item Behavioual Shift in Knowledge Work
    \item Integrated Conceptual Framework
  \end{enumerate}
  \item Comparative Analysis: Shadow IT vs. Shadow AI
  \item Discussion
  \item Limitations and Future Research Directions
  \item Conclusion
\end{enumerate}

With this outline, the thesis will first establish a theoretical foundation by discussing the concepts of Shadow IT and motivational theories. Then, it will present the methodology of the systematic literature review and its results. Based on these results, the thesis will develop a conceptual framework that extends the traditional understanding of Shadow IT to encompass the unique characteristics and implications of Shadow AI. Finally, a comparative analysis will be conducted to highlight the differences and similarities between Shadow IT and Shadow AI, followed by a discussion of the findings, limitations, and future research directions.

%%%%%%%%%%%%%%%%%%%%%%%%%%%%
%% Literaturverzeichnis wird 
%% automatisch eingefügt
%%%%%%%%%%%%%%%%%%%%%%%%%%%%
\clearpage
\lhead{}
\printbibliography
\addcontentsline{toc}{section}{\bibname}


%%%%%%%%%%%%%%%%%%%%%%%%%%%%
%% Anhang (optional) 
%%%%%%%%%%%%%%%%%%%%%%%%%%%%
%\clearpage
%\appendix
%\section{Appendix A} % Bei englischsprachiger Arbeit anzupassen auf: Appendix A

%%%%%%%%%%%%%%%%%%%%%%%%%%%%
%% Eidesstattliche Erklärung
%% muss angepasst werden 
%% in Erklaerung.tex
%%%%%%%%%%%%%%%%%%%%%%%%%%%%
\input{Erklaerung.tex}

\end{document}