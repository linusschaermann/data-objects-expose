\documentclass[12pt,oneside]{article}

%%%%%%%%%%%%%%%%%%%%%%%%%%%%
%%   Zusaetzliche Pakete  %%
%%%%%%%%%%%%%%%%%%%%%%%%%%%%
% use enumitem (do NOT load the older 'enumerate' package together with enumitem)
\usepackage{fancyhdr}
\usepackage{a4wide}
\usepackage{graphicx}
\usepackage{palatino}
\usepackage{multirow}
\usepackage{booktabs}
\usepackage{titlesec}
\usepackage{acronym}% http://ctan.org/pkg/acronym
\usepackage{enumitem}% http://ctan.org/pkg/enumitem

%folgende Zeile auskommentieren für englische Arbeiten
%\usepackage[ngerman]{babel}
%folgende Zeile auskommentieren für deutsche Arbeiten
\usepackage[ngerman, english]{babel}

\usepackage[T1]{fontenc}
\usepackage[utf8]{inputenc}
\usepackage[bookmarks]{hyperref}
\usepackage[justification=centering]{caption}
\usepackage[style=apa,natbib=true,backend=biber,maxbibnames=20]{biblatex}
\usepackage{csquotes}
\bibliography{literature}

\setlength{\parindent}{0em} 
\setlist[itemize]{noitemsep, topsep=0pt}
\setlist[enumerate]{noitemsep, topsep=0pt}
% Use numeric labels for enumerate levels:
\setlist[enumerate,1]{label=\arabic*.}
\setlist[enumerate,2]{label=\arabic{enumi}.\arabic{enumii}.}
\setlist[enumerate,3]{label=\arabic{enumi}.\arabic{enumii}.\arabic{enumiii}.}


%%%%%%%%%%%%%%%%%%%%%%%%%%%%%%
%% Definition der Kopfzeile %%
%%%%%%%%%%%%%%%%%%%%%%%%%%%%%%

\pagestyle{fancy}
\fancyhf{}
\cfoot{\thepage}
\setlength{\headheight}{16pt}

%%%%%%%%%%%%%%%%%%%%%%%%%%%%%%%%%%%%%%%%%%%%%%%%%%%%%
%%  Definition des Deckblattes und der Titelseite  %%
%%%%%%%%%%%%%%%%%%%%%%%%%%%%%%%%%%%%%%%%%%%%%%%%%%%%%

\newcommand{\JMUTitle}[9]{

  \thispagestyle{empty}
  \vspace*{\stretch{1}}
  {\parindent0cm
  \rule{\linewidth}{.7ex}}
  \begin{flushright}
    \vspace*{\stretch{1}}
    \sffamily\bfseries\Huge
    #1\\
    \vspace*{\stretch{1}}
    \sffamily\bfseries\large
    #2\\
    \vspace*{\stretch{1}}
    \sffamily\bfseries\small
    #3
  \end{flushright}
  \rule{\linewidth}{.7ex}

  \vspace*{\stretch{1}}
  \begin{center}
    \includegraphics[width=2in]{siegel} \\
    \vspace*{\stretch{1}}
    \Large Exposé  \\

    \vspace*{\stretch{2}}
   \large Lehrstuhl f\"{u}r Wirtschaftsinformatik\\
    \large und Systementwicklung\\
    \large Universität Würzburg\\
    \vspace*{\stretch{1}}
    \large Betreuer:  #8 \\[1mm]
    
    \vspace*{\stretch{1}}
    \large W\"urzburg, den #7 \\
        \vspace*{\stretch{0.25}}

    % Bearbeitungszeit: 14.03.2025 - 09.05.2025 % Die Bearbeitungszeit der Seminar-/ Abschlussarbeit ist hier einzutragen.

  \end{center}
}

\titlespacing*{\section}
{0pt}{3.5ex plus 1ex minus .2ex}{.2ex plus .2ex}
\titlespacing*{\subsection}
{0pt}{1.5ex plus 1ex minus .2ex}{.2ex plus .2ex}
\titlespacing*{\subsubsection}
{0pt}{1.5ex plus 1ex minus .2ex}{.2ex plus .2ex}




%%%%%%%%%%%%%%%%%%%%%%%%%%%%
%%  Beginn des Dokuments  %%
%%%%%%%%%%%%%%%%%%%%%%%%%%%%

\begin{document}

  \JMUTitle
      {Data Objects as epistemic and managerial boundary objects – how organizations produce, interpret, and act upon data objects in machine learning contexts.}        % Titel der Arbeit
      {Linus Schärmann}                        % Vor- und Nachname des Autors
      {2910412}
      
      {Wirtschaftswissenschaftlichen Fakultät}  % Name der Fakultaet
      {W"urzburg 2025}                          % Ort und Jahr der Erstellung
      {08.11.2025}                              % Tag der Abgabe
      {Tim Thorwart-Gumpert}               % Name des Erstgutachters
      {}                          % Name des Zweitgutachters

  \clearpage

\lhead{}
\pagenumbering{Roman} 
    \setcounter{page}{1}

\tableofcontents
\clearpage

%%%%%%%%%%%%%%%%%%%%%%%%%%%%
%%  Kurzzusammenfassung   %%
%%%%%%%%%%%%%%%%%%%%%%%%%%%%
% \newpage
% \lhead{Abstract}
% \section*{Abstract}
% \addcontentsline{toc}{section}{Abstract}

%\newpage
%\lhead{List of Figures} % Bei englischsprachiger Arbeit anzupassen auf: List of Figures
%\addcontentsline{toc}{section}{List of Figures} % Bei englischsprachiger Arbeit anzupassen auf: List of Figures
%\listoffigures

% \newpage
% \lhead{List of Tables} % Bei englischsprachiger Arbeit anzupassen auf: List of Tables
% \addcontentsline{toc}{section}{List of Tables} % Bei englischsprachiger Arbeit anzupassen auf: List of Tables
% \listoftables
% \newpage

\setlength{\parskip}{0.5em} 


%%%%%%%%%%%%%%%%%%%%%%%%%%%%%%%%%%
%%  Definition der Abkürzungen  %%
%%%%%%%%%%%%%%%%%%%%%%%%%%%%%%%%%%
\lhead{List of Abbreviations} % Bei englischsprachiger Arbeit anzupassen auf: List of Abbreviations
\section*{List of Abbreviations} % Bei englischsprachiger Arbeit anzupassen auf: List of Abbreviations
\addcontentsline{toc}{section}{List of Abbreviations} % Bei englischsprachiger Arbeit anzupassen auf: List of Abbreviations

\begin{acronym}
  \acro{is}[IS]{Information Systems}
  \acro{slr}[SLR]{Systematic Literature Review}
\end{acronym}

%%%%%%%%%%%%%%%%%%%%%%%%%%%%
%%  Einstellungen  %%
%%%%%%%%%%%%%%%%%%%%%%%%%%%%
\clearpage
\pagenumbering{arabic}  
    \setcounter{page}{1}
\lhead{\nouppercase{\leftmark}}

%%%%%%%%%%%%%%%%%%%%%%%%%%%%
%%  Hauptteil  %%
%%%%%%%%%%%%%%%%%%%%%%%%%%%%

\section{Research Topic} \label{researchtopic}
Short introduction, explanation of the topic and its relevance. 

\section{Motivation} \label{motivation}
Explain what you want to find out by the end of your research project—and why!

\section{Methodology} \label{methodology}

The research will follow a \ac{slr} according to \citet{RN2}.
This method is used to identify, evaluate and interpret the available research relevant to a particular research question, topic area or domain of interest and is designed to be used in \ac{is} research \citep[884]{RN2}. 
\newline
A \ac{slr} is particularly suitable for this thesis as it allows to systematically gather and analyze existing knowledge on the topic of data objects in machine learning contexts in general and above all, organizational contexts.
With this approach, it is possible to identify research gaps and provide a solid foundation for future research directions.
\citet{RN2}'s guidelines include the following steps in order to follow a structured approach:


\section{Preliminary Outline}

Lastly the following outline should give an overview of the planned sections of the thesis. Changes may appear during the research process but as an entry point it should help to structure the work.

\begin{enumerate}
  \item Introduction
  \item Theoretical Background
  \begin{enumerate}
    \item Boundary Objects
    \item Data Objects as Epistemic Artifacts
  \end{enumerate}
  \item Methodology
  \item Structured Literature Review Results
  \begin{enumerate}
    \item Overview of Identified Literature
    \item Technical Origin of Data Objects in ML Contexts
    \item Data Objects as Boundary Objects in Organizational Contexts
    \item Visualization and Interface Design as Epistemic Factors
  \end{enumerate}
  \item Discussion
  \item Future Research Directions
  \item Conclusion
\end{enumerate}


%%%%%%%%%%%%%%%%%%%%%%%%%%%%
%% Literaturverzeichnis wird 
%% automatisch eingefügt
%%%%%%%%%%%%%%%%%%%%%%%%%%%%
\clearpage
\lhead{}
\printbibliography
\addcontentsline{toc}{section}{\bibname}


%%%%%%%%%%%%%%%%%%%%%%%%%%%%
%% Anhang (optional) 
%%%%%%%%%%%%%%%%%%%%%%%%%%%%
%\clearpage
%\appendix
%\section{Appendix A} % Bei englischsprachiger Arbeit anzupassen auf: Appendix A

%%%%%%%%%%%%%%%%%%%%%%%%%%%%
%% Eidesstattliche Erklärung
%% muss angepasst werden 
%% in Erklaerung.tex
%%%%%%%%%%%%%%%%%%%%%%%%%%%%
\input{Erklaerung.tex}

\end{document}